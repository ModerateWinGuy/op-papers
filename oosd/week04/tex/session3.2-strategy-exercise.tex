\documentclass{article}
\usepackage{graphicx}
\usepackage{wrapfig}
\usepackage{inconsolata}
\usepackage{enumerate}
\usepackage{hyperref}
\usepackage[margin = 2.25cm]{geometry}



\begin{document}

\begin{figure}
\includegraphics[width=30mm]{../../../resources/images/oplogo.png}
\end{figure}

\title{Design Pattern Exercise\\IN710 Object Oriented System Development}
\date{}
\maketitle

\section*{Introduction}
In this exercise you reuse your \texttt{Card} and \texttt{Deck} classes.  
You will modify \texttt{Deck} to use the \emph{Strategy} 
pattern for shuffling and then measure the effectiveness of different 
shuffling methods.

\section{Shuffling}
In your \texttt{Deck} class you have a list of \texttt{Card}s, and at some 
point you need to shuffle them.  There are a number of ways to do this.

\begin{enumerate}
	\item by using the \texttt{random.shuffle()} function;
	\item exchange each card in the deck with one at a 
		randomly selected index;
	\item for each evenly numbered card position, exchange
		it with a randomly chosen odd position;
	\item swap two randomly chosen card positions 52 or more times.
\end{enumerate}

You can probably come up with others.

\section{Task 1}
Implement the various shuffling algorithms as \emph{strategies} that can be
plugged into your \texttt{Deck} class.  Then write a simple program that
creates one deck for each strategy and applies the appropriate shuffle.

\section{Task 2}
Now we want to test teh effectiveness of each shuffling method.  To do this,
we will shuffle the decks 520 times with each shuffling method and record the results.
If a shuffling method is perfect, then every card should occur in every deck postion
10 times.

To record the results, make a Python dictionary using the card values as keys and whose
values are lists showing the positions (0 - 51) that a card occupies in each shuffled deck.

We can use a chi-squared test to evaluate the quality of each shuffle.  The general 
chi-squared formula is

\vspace{5mm}
$\chi^2 = \sum_{i=0}^{n} (a_{i} - e_{i})^2/e_{i}$
\vspace{5mm}

Where $a_{i}$ is the observed card position frequency  and $e_{i}$ is the expected
frequency (10 in our case).

Write a chi-squared function that takes a results dictionary and ruturns the 
corresponding value.

\end{document}
